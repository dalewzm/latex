\section{绪论}
正文不少于10千字;或使用小四字体、1.5倍行距、A4纸版式排版时不少于10页纸。正文须有页码,从第1页开始编页码。正文采用章、节、小节组织。章的标题使用“第一章”等字样开头,节的标题采用“1.1”等字样开头,表示第一章的第一节,小节的标题采用“1.1.1”等字样开头,表示第一章的第1.1小节。正文章、节、小节标题与正文段落使用不同的字体,并且之间有适当的间距。正文段落要统一缩进两个汉字。
行文时注意语句通顺,条理清晰;每章节开头部分需要有承上启下描述,先简要介绍本章节内容,再展开详细描述。
第一章作为概述,也是完整的短文,体现全文的内容
\subsection{研究背景}
阐明问题的来源、研究的动机、意义等
\subsection{研究现状}
\subsection{本文工作}
阐述本文的主要工作,即简要描述提出的方法、创新点、结果。
\subsection{论文章节安排}
简单介绍论文后面章节的安排和主要内容