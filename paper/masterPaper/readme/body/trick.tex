\chapter{Latex中的一些技巧}

\tcbset{demobox/.style={colback=black!10}]}

\section{版面设置}
中文字号设置\\
\begin{tabular}{ccc}  
\toprule[1.5pt]  
命令 & 大小& 实际显示 \\  
\midrule  
\textbackslash zihao\{ 0  \} & 42& {\zihao{0} 初号} \\  
\textbackslash zihao\{ -0  \} & 36& {\zihao{-0} 小初号} \\  
\textbackslash zihao\{ 3  \} & 16& {\zihao{3} 三号} \\ 

...&...&...\\  
\textbackslash zihao\{ -4  \} & 12& {\zihao{-4} 小四号} \\ 
\bottomrule[1.5pt]  
\end{tabular}  


\section{数学环境}
公式问题

\section{源码编写}
注释:
 单行注释
使用\%号
\begin{tkzexample}[code only,small,num]
这一行将不会被编译
haha
\end{tkzexample}
 多行注释
使用$\backslash$ iffalse $\backslash$ fi结构
\begin{tkzexample}[code only,small,num]
下面文字被注释了
\iffalse
我不信
我没有被注释
\fi
请看效果
\end{tkzexample}
\tcbset{colback=black!10}
\begin{tcolorbox}
下面文字被注释了
\iffalse
我不信
我没有被注释
\fi
请看效果
\end{tcolorbox}


\section{编写其它类型文本所需}
旁白实现:

使用marginpar命令可以给文档加上边注
\begin{tkzexample}[code only,small,num]
测试旁白效果 \marginpar{我就是旁白}
\end{tkzexample}
测试旁白效果 \marginpar{我就是旁白}

\begin{tcolorbox}[title=演示效果]
$$ \text{中文公式}a^2+b^2=c^2 $$
\end{tcolorbox}

\tcbset{frogbox/.style={enhanced,colback=green!10,colframe=green!65!black,
enlarge top by=5.5mm,
overlay={\foreach \x in {2cm,3.5cm} {
\begin{scope}[shift={([xshift=\x]frame.north west)}]
\path[draw=green!65!black,fill=green!10,line width=1mm] (0,0) arc (0:180:5mm);
\path[fill=black](-0.2,0) arc (0:180:1mm);
\end{scope}}}]}}




\begin{tcolorbox}[frogbox,title=My title]
This is a \textbf{tcolorbox}.
\end{tcolorbox}


 有时候要在Latex中插入代码段,如果直接作为正文插入,则会跟Latex的很多关键字冲突,那么需要用到Verbatim环境,


